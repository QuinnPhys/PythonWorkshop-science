%!TEX program=xelatex
% This cheatsheet is based on the template
% provided at https://gist.github.com/alexander-yakushev/c773543bf9a957749f79.
\documentclass[10pt,english,landscape]{article}
\usepackage{multicol}
\usepackage{calc}
\usepackage[landscape]{geometry}
\usepackage{color,graphicx,overpic}

% https://tex.stackexchange.com/a/192067
\usepackage{fontawesome}
\usepackage[otf]{sourcecodepro}
  % \setmonofont{Source Code Pro}
% \newcommand{\faWindows}{\FA\symbol{"F17A}}
% \newcommand{\faLinux}{\FA\symbol{"F17C}}
% \newcommand{\faApple}{\FA\symbol{"F179}}

% \usepackage[T1]{fontenc}
% \usepackage[bitstream-charter]{mathdesign}
\usepackage[utf8]{inputenc}
\usepackage{url}
\usepackage{amsfonts}
\usepackage{array,booktabs}
\usepackage{textcomp}
\usepackage[usenames,dvipsnames,table]{xcolor}
\usepackage[most]{tcolorbox}
\usepackage{tabularx}
\usepackage{multirow}
\usepackage{colortbl}
\usepackage{tikz}
\usepackage{environ}

\usetikzlibrary{calc}
\pgfdeclarelayer{background}
\pgfdeclarelayer{foreground}
\pgfsetlayers{background,main,foreground}

\geometry{top=-0.5cm,left=1cm,right=1cm,bottom=1cm}

\pagestyle{empty} % Turn off header and footer

% \renewcommand\rmdefault{phv} % Arial
% \renewcommand\sfdefault{phv} % Arial

% Redefine section commands to use less space
\makeatletter
\renewcommand{\section}{\@startsection{section}{1}{0mm}%
  {-1ex plus -.5ex minus -.2ex}%
  {0.5ex plus .2ex}%x
  {\normalfont\large\bfseries}}
\renewcommand{\subsection}{\@startsection{subsection}{2}{0mm}%
  {-1explus -.5ex minus -.2ex}%
  {0.5ex plus .2ex}%
  {\normalfont\normalsize\bfseries}}
\renewcommand{\subsubsection}{\@startsection{subsubsection}{3}{0mm}%
  {-1ex plus -.5ex minus -.2ex}%
  {1ex plus .2ex}%
  {\normalfont\small\bfseries}}
\makeatother

\setcounter{secnumdepth}{0} % Don't print section numbers
\setlength{\parindent}{0pt}
\setlength{\parskip}{0pt plus 0.5ex}

\definecolor{TableHead}{rgb}{0.353, 0.329, 0.667}
\definecolor{TableRow}{rgb}{0.816, 0.812, 0.902}

\NewEnviron{keys}[1]{
  % \begin{center}
  \smallskip
  \begin{tikzpicture}
      \rowcolors{1}{}{TableRow}
      \centering
      \node (tbl) [inner sep=0pt] {
        \begin{tabular}{p{2.9cm}p{4.65cm}}
          \rowcolor{TableHead}
          \multicolumn{2}{l}{\normalsize\textbf{\color{white}{#1}}}\parbox{0pt}{\rule{0pt}{0.3ex+\baselineskip}}\\
          \BODY
          \arrayrulecolor{TableHead}\specialrule{.17em}{0em}{.2em}
        \end{tabular}};
      \begin{pgfonlayer}{background}
        \draw[rounded corners=2pt,top color=TableHead,bottom color=TableHead, draw=white]
        ($(tbl.north west)-(0,-0.05)$) rectangle ($(tbl.north east)-(0.0,0.15)$);
        \draw[rounded corners=2pt,top color=TableHead,bottom color=TableHead, draw=white]
        ($(tbl.south west)-(0.0,-0.11)$) rectangle ($(tbl.south east)-(-0.0,-0.02)$);
      \end{pgfonlayer}
    \end{tikzpicture}
  % \end{center}
}

% https://tex.stackexchange.com/a/102523
\newcommand{\forceindent}[1]{\leavevmode{\parindent=#1\indent}}

\begin{document}

\raggedright\

\begin{center}
  \Large{\underline{Python Workshop Cheatsheet}}
\end{center}

\footnotesize
\begin{multicols}{3}

  \centering\section{Command Line}

  \begin{keys}{Basics}
    Change directory    & \texttt{cd \emph{dirname}} \\
    Go to home & \texttt{cd \textasciitilde} \\
    Go up one dir. & \texttt{cd ..} \\
    List files          & \texttt{ls} \\
    List files (incl hidden) & \texttt{ls -a} \\
    Print working dir. & \texttt{pwd} \\
    Make new directory & \texttt{mkdir \emph{dirname}} \\
    Remove empty dir. & \texttt{rmdir} \\
    Remove files & \texttt{rm \emph{filename}} \\
    Remove files and\newline directories & \texttt{rm -r \emph{filename}}\hfill\faWarning \\
    Find path for\newline command & \texttt{which \emph{cmd}}\hfill\faLinux \faApple \\
                          & \texttt{Get-Command \emph{cmd}}\hfill\faWindows \\
    Show file contents & \texttt{cat \emph{file}} \\
    Show long file & \texttt{less \emph{file}}\hfill\faLinux\faApple \\
    & \texttt{more \emph{file}}\hfill\faWindows \\
    Show manual & \texttt{man \emph{cmd}} \\
    Show online help & \texttt{Get-Help -Online \emph{cmd}}\hfill\faWindows \\
    Write to file & \texttt{echo ''\emph{contents}'' > \emph{file}}\hfill\faLinux \faApple \\
                  & \texttt{sc \emph{file} -Value ''\emph{contents}''}\hfill\faWindows \\
  \end{keys}

  % \begin{keys}{SSH}
  %   a & \texttt{todo} \\
  % \end{keys}

  \begin{keys}{Git}
    Clone repo & \texttt{git clone \emph{url}} \\
    Show status & \texttt{git status} \\
    Stage file & \texttt{git add \emph{file}} \\
    Unstage file & \texttt{git reset \emph{file}} \\
    Commit staged\newline changes & \texttt{git commit -m ''\emph{message}''} \\
    Change branch & \texttt{git checkout \emph{branch}} \\
    New branch & \texttt{git checkout -b \emph{branch}} \\
    Push current branch & \texttt{git push} \\
    Pull current branch & \texttt{git pull} \\
    Discard changes in file & \texttt{git checkout -{}- \emph{file}} \\
    Stash current changes & \texttt{git stash} \\
    Apply previous stash & \texttt{git stash apply} \\
  \end{keys}

  \begin{keys}{Package Management}
    Install OS pkg & \texttt{sudo apt-get install \emph{pkg}}\hfill\faLinux \\
                       & \texttt{choco install \emph{pkg}}\hfill\faWindows \\
    Install Anaconda pkg & \texttt{conda install \emph{pkg}} \\
    Install Python pkg & \texttt{pip install \emph{pkg}} \\
  \end{keys}

  \columnbreak%\

  \centering\section{Python Language}

  \begin{keys}{Built-in Types}
    Strings                            & \texttt{str} \\
    Numeric types                      & \texttt{int} and \texttt{float} \\
    Boolean                            & \texttt{bool}, \texttt{True} and \texttt{False} \\
    Lists                              & \texttt{list} and \texttt{[]} \\
    Tuples                             & \texttt{tuple} and \texttt{()} \\
    Dictionaries                       & \texttt{dict} and \texttt{\{\}} \\
  \end{keys}

  \begin{keys}{Definitions}
    Functions & \texttt{def \emph{fn}(\emph{arg}, \emph{kwarg}=\emph{default}):}\newline
                \texttt{\forceindent{2em}\emph{body}} \\
    Classes & \texttt{class \emph{Name}(\emph{bases}):}\newline
              \texttt{\forceindent{2em}\emph{body}} \\
  \end{keys}

  \begin{keys}{Flow Control}
    Conditional                        & \texttt{if \emph{cond}:}\newline\texttt{\forceindent{2em}\emph{body}} \\
    For-each loop                      & \texttt{for \emph{var} in \emph{iterable}:}\newline
                                         \texttt{\forceindent{2em}\emph{body}} \\
    Do-while                           & \texttt{while \emph{cond}:}\newline\texttt{\forceindent{2em}\emph{body}} \\
    Context manager                    & \texttt{with \emph{manager} as \emph{var}:}\newline\texttt{\forceindent{2em}\emph{body}} \\
  \end{keys}

  \begin{keys}{Comprehensions}
    List comp. & \texttt{[\emph{expr} for \emph{var} in \emph{iterable}]} \\
    Dict. comp. & \texttt{\{\emph{k}: \emph{v} for \emph{var} in \emph{iterable}\}} \\
  \end{keys}

  \begin{keys}{Iterator Examples}
    \texttt{range} & \texttt{list(range(3))}\newline~$\to$\texttt{[0, 1, 2]} \\
    \texttt{zip} & \texttt{list(zip("ab", "AB"))}\newline~$\to$ \texttt{[("a", "A"), ("b", "B")]} \\
    \texttt{enumerate} & \texttt{list(enumerate("ab"))}\newline~$\to$ \texttt{[(0, "a"), (1, "b")]} \\
    \texttt{itertools.product} & \texttt{list(product(range(2), "AB"))}
                                 \newline~$\to$\texttt{[(0, "A"), (0, "B"), \dots]}\\
  \end{keys}

  \begin{keys}{PEP8 Names}
    Functions, variables, etc. & \texttt{snake\_case} \\
    Clases, types, etc. & \texttt{CamelCase} \\
    Constants & \texttt{SHOUTY\_CASE} \\
  \end{keys}

  \columnbreak%\

  \centering\section{NumPy}

  \begin{keys}{\texttt{import numpy as np}}
    New array & \texttt{np.array([\emph{\dots}])} \\
    Index     & \texttt{\emph{arr}[0]} \hfill(1D, scalar)\newline
                \texttt{\emph{arr}[\emph{start}:\emph{end}]} \hfill(1D, slice)\newline
                \texttt{\emph{arr}[\emph{ax0}, \emph{ax1}, \dots]}\hfill(multidim)\newline
                \texttt{\emph{arr}[np.newaxis]}\hfill(new axis) \\
    Transpose & \texttt{\emph{arr}.T} or \newline
                \texttt{\emph{arr}.transpose(\emph{axes})} \\
    Reshape   & \texttt{\emph{arr}.reshape(\emph{new\_shape})} \\
    Reduce    & \texttt{\emph{arr}.sum(axis=\emph{axes})} or \newline
                \texttt{\emph{arr}.mean(axis=\emph{axes})}, etc. \\
  \end{keys}

  \centering\section{Plotting}

  \begin{keys}{\texttt{import matplotlib.pyplot as plt}}
    Plot & \texttt{plt.plot(\emph{xs}, \emph{ys}, '\emph{style}', \newline\forceindent{1em}label=\emph{label})} \\
    Axis labels & \texttt{plt.xlabel(\dots)} and \texttt{plt.ylabel(\dots)} \\
    Axis limits & \texttt{plt.xlim}  and \texttt{plt.ylim} \\
    Plot title & \texttt{plt.title(\dots)} \\
    Legend & \texttt{plt.legend()} \\
  \end{keys}


  \centering\section{IPython / Jupyter}

  \begin{keys}{Magic Commands}
    Meas. execution\newline time & \texttt{\%timeit \emph{stmt}} \\
    Navigate dirs. & \texttt{\%cd \emph{dirname}, \%pwd, \%ls} \\
    Inline plotting (NB) & \texttt{\%matplotlib inline} \\
    Help on symbol & \texttt{\emph{name}?} \hfill(short)\newline \texttt{\emph{name}??} \hfill(complete) \\
  \end{keys}

  \begin{keys}{Notebook Shortcut Keys}
    Run current cell & \textbf{Shift + Enter} \\
    New cell above & \textbf{Esc}, \textbf{A} \\
    New cell below & \textbf{Esc}, \textbf{B} \\
    Delete current cell & \textbf{Esc}, \textbf{D}, \textbf{D} \\
    Change cell to\newline Markdown & \textbf{Esc}, \textbf{M} \\
    Comment/uncomment & \textbf{Ctrl + /} \\
    Show all shortcuts & \textbf{Esc}, \textbf{H} \\
  \end{keys}

\end{multicols}

\end{document}